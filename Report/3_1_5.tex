%!TEX root = 0_architecture_rapport.tex

% subsection title
\subsection{Question 5}
\label{subsec:315}
\paragraph{}We thought that some of the highest salaries may be explained by a phenomenon of exaggerated enthusiasm for specific players who are playing well during the regular season but either cannot get their team to the playoffs or perform poorly once in the playoffs. To investigate this hypothesis, we decided to scrap the playoff statistics, that is to say statistics of teams that reached the final part of the competition. We thought that, by focusing on the best teams for each season, we would be able to assess the players who perform well when it really matters (in the playoffs). These data were scraped on the page \url{http://www.basketball-reference.com/playoffs/} with the same technique as in \ref{subsec:312} and \ref{subsec:314}, since they are also stored in a well-structured table on the website. 

Our recommendation to team owners is to consider players that appear in this page differently than other players. Performance during the playoffs should be more valued than performance during the regular season. Also the owners should pay attention to discrepancies between regular season production and playoffs production. It would help them not to pay salaries that are not deserved (to player performing worse during the playoffs) and use this money for more efficient players (who perform well in the playoffs).
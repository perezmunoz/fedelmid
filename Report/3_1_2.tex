%!TEX root = 0_architecture_rapport.tex

% subsection title
\subsection{Question 2}
\label{subsec:312}
We decided to scrap the pages of all the players and not only the active ones, as the data could be useful to assess the salaries of the players. However, we also created a function \verb|active_players_BS| which return which players are active and which are not.

The players are classified according to the first letter of their last name. We first retrieve all the letters which have players whose name starts with it, from \url{http://www.basketball-reference.com/players}. For example, there is no player whose name starts with $x$. For a letter $c$, we then download the page \url{http://www.basketball-reference.com/players/c} and extract the player's page links.

% subsubsection title
\subsubsection{Question 2.a}
\label{subsubsec:312a}
\paragraph{} The regular expression we use to retrieve the letters is:

\begin{minted}[frame=single,linenos,mathescape,fontsize=\small]{html}
<a href="/players/([a-z]+)/">[A-Z]+</a></td>
\end{minted}

%\mint{html}|<a href="/players/([a-z]+)/">[A-Z]+</a></td>|

The regular expression we use to identify the links to the player's page is the following:

\begin{minted}[frame=single,linenos,mathescape,fontsize=\small]{html}
[^p]><a href="(/players/./.+)"
\end{minted}

%\mint{html}|[^p]><a href="(/players/./.+)"|
% subsubsection title
\subsubsection{Question 2.b}
\label{subsubsec:312b}

\paragraph{}The code we use to retrieve the letters with BeautifulSoup4 is:

\begin{minted}[frame=single,linenos,mathescape,fontsize=\small]{python}
for row in soup('td', {'class': 'align_center bold_text valign_bottom
								 xx_large_text'}):
	letter = str(row.a.get('href').split('/')[2])
\end{minted}

\paragraph{}The code we use to identify the links with BeautifulSoup4 is the following:

\begin{minted}[frame=single,linenos,mathescape,fontsize=\small]{python}
for player in soup.tbody.find_all('tr'):
	link = str(player.td.a.get('href'))
\end{minted}

\paragraph{}The whole can be found at \url{Code/Part1/BS} and \url{Code/Part1/Regex} for the BeautifulSoup4 and Regex respectively.

% subsubsection title
\subsubsection{Question 2.c}
\label{subsubsec:312c}
\paragraph{} We chose to use the regex code because it is faster and use less memory. In fact, we ran 100 instances of both versions and obtained the following results:
\begin{center}
	\begin{tabular}{| c | c | c |}
	\hline
	Method & Time & Memory size \\ \hline
	regex 	& 1.167 s & 2.6 MB \\ \hline
	BS & 3.911 s & 3.7 MB \\
	\hline
	\end{tabular}
\end{center}

Those results were obtained using pre-downloaded pages to avoid measuring the download times stored in \url{Data/Part1/Metrics/Pages}.

Furthermore, the regex code is more readable since it contains less functions that need to be called.
